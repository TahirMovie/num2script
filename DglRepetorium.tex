\documentclass[11pt]{article}
\usepackage{amsfonts}
\usepackage{ulem}
\usepackage{amsmath}
\usepackage{amsthm}
\usepackage{mathtools}
\usepackage{pstricks}
\usepackage{enumitem}
\usepackage[a4paper]{geometry}
\usepackage{fancyhdr}
\usepackage{titlesec}
\usepackage{amssymb}
\usepackage{hyperref}

\newcommand{\dd}{\mathrm{d}}
\newcommand{\ee}{\mathrm{e}}
\newcommand{\RR}{\mathbb{R}}
\newcommand{\NN}{\mathbb{N}}
\newcommand{\UU}{\mathcal{U}}

\newcounter{myCounter}[section]
\newtheorem{Def}[myCounter]{Definition}
\newtheorem{Bem}[myCounter]{Bemerkung}
\newtheorem{Bsp}[myCounter]{Beispiel}
\newtheorem{Satz}[myCounter]{Satz}

\title{Repetorium der Gew\"ohlichen Differenzialgleichung}
\author{}

\begin{document}

\maketitle

\begin{Def}
  Als gew\"ohnliche Differenzialgleichung bezeichent man eine Gleichung, in der eine (unbekannte)
  Funktion $u: I \subset \RR \rightarrow \UU \subset \RR^m , t \mapsto w(t)=
  (u_1(t), \dotsc, u_m(t))$ und ihre Abbildung bis zu einer gewissen Ordnung $
  k \in \NN$ auftauchen. \[
    f(t,u(t), u'(t), \dotsc, u^{(k)}(t)) = 0 \]
  mit \[
    f:I \times \RR^{(k+1)m} \rightarrow \RR^m \]
  gegeben sind.
\end{Def}

\begin{Bem}
  ``L\"osung'' bedeutet: $u\in C^k(I,\RR^m)$
  \begin{itemize}
  \item f\"ur $n > 1$ spricht man von einem System
  \item Gew\"ohnliche Differenzialgleichung der Form \[
    u^{(k)}(t) = \tilde{f}(t,u(t),\dotsc,u^{(k-1)}(t)) \]
    nennt man \emph{explizite} Differenzialgleichung.
  \end{itemize}
\end{Bem}
Wir betrachten ein System gew\"ohnlicher Differenzialgleichung \emph{erster Ordnung}, d.h.\[
  \left\{\begin{aligned} u(t) &= f(t,u(t0)), \quad t \in I \\ u(t_0) &= u_0
  \end{aligned} \right. \]
ein Anfangswertproblem zum Anfangswert $u_0$.


Eine Allgemeine Differenzialgleichung mit $y(t) = \RR$ \[
  y^{(k)}(t) = F(t,\overbrace{y(t),\dotsc,y^{(k-1)}(t)}^{=u(t)}) \]
l\"asst sicht immer auf eine Differenzialgleichung erster Ordnung zur\"uckf\"uhren, denn f\"ur
\[ u_1(t):=y(t), u_2(t) := y'(t), \dotsc, u_{k-1}(t)=y^{(k-1)}(t) \] folgt \[
  u'(t) = \begin{pmatrix} u_2(t) \\ \vdots \\ u_k(t) \\ F(t,u(t) \end{pmatrix} =
: f(t,u(t))) \]

\begin{Bsp}
\[ z''(t) + z'(t) + (z(t)t^2)^2 = z'''(t) \tag{1} \]
Definieren $u(t) = z(t), z'(t), z''(t) )$ \[
  \Rightarrow u'(t) = \begin{pmatrix} z' \\ z'' \\ z''' \end{pmatrix}  
  \mathop{=}^{(1)} \begin{pmatrix} z'(t) \\ z''(t) \\ z'' + z' + (z + t^2)^2
  \end{pmatrix} = f(t,u(t)) \]
\end{Bsp}

\section{Existenz und Eindeutigkeit von L\"osungen}

\paragraph{Picard-Lindel\"of}
Betrachten $e'(t) = f(t,u(t)), u(0) = u_0$, $f$ stetig und erf\"ulle die glaobale
Lipschizbedingung \[ \exists_{L>0} \forall_{(t,u),(t,v)} : f(t,u) - f(t,v) \leq
|u-v| \] dann gibt es eine L\"osung $u \in C^1$.

\paragraph{Picard Interationsverfahren}
Im Beweis des Satzes von Picard-Lindel\"of wird der Banachsche Fixpunktsatz
benutzt \[ (Tu)(t) = u_0 + \int_0^t f(s,u(s)) \dd s. \]
Der Banachsche Fixpunktsatz ist Konstuktiv, d.h. f\"ur $u_0$ \emph{beliebig}
und $u_{l+1} = Tu_l$ konvergiert die Folge $(u_l)_{l \in \NN}$ gegen die
L\"osung $u(t)$.

\begin{Bsp}
  \[ \left\{\begin{aligned} u(t)&=t+ u(t) \\ u(0) &= 0 \end{aligned} \right.
    \text{ oder } f(v_1,v_2) = v_1 + v_2 \]
  \paragraph{Frage} existiert eine eindeutige L\"osung? Falls ja, geben Sie die
  L\"osung explizit an.

  Es gilt: $f(t,u) - f(t,v) = 1 (u - v) \rightarrow \text{ Vorraussetzunen von
  Picard-Linde\"of erf\"ullt d.h. es gibt eindeutige L\"osung }$

  Betrachte Iterationsfolge \[
    u_{l+1} - u_0 + \int_0^t f(s,u_l) \dd s \]
    W\"ahle $u_ 0 = 0$ , dann folgt
  \begin{align*}
    \begin{matrix}
      u_1(t) &= 0 + \int_0^t s + 0 \dd s = \frac{t^2}{2} \\
      u_2(t) &= 0 + \int_0^t s + \frac{s^2}{2} \dd s = \frac{t^2}{2} + \frac{t^3}{3!} \\
      \vdots \\
      u_m(t) &= \frac{t^2}{2} + \frac{t^3}{3!} + \dotsb + \frac{t^{m+1}}{(m=1)!}
    \end{matrix}
  \end{align*}

  Somit \[ u(t) = e^t - t - 1 \quad \text {e-Reihe!} \]

  \paragraph{Probe} $u(t0) = 0 \ u'(t) = e^t - 1 = e^t - t - 1 + t = t +
  u(t)$
\end{Bsp}

\section{Lineare autonome Systeme von Differenzialgleichungn}

\begin{Def}
  Eine autonome (lineare) Differenzialgleichung ist eine explizite Differenzialgleichung, deren rechte Seite nicht
  explizit von der unabh\"anigen Variable abh\"angt. \[
    u'(t) = A u(t) \]
  wobe  $A \in \RR^{m \times m}$ und $u(t) \in \RR^m$
\end{Def}

Fundamentalmatrix $Y_0(t) = e^{tA}$ gegeben nach Vorlesung.

\begin{Bsp}
  \[ \left\{ \begin{aligned} u'_1(t) = 2 u_2(t) + u_1(t) \\ u'_2(t) = 3 u_2(t) = 4
  u_1(t) \end{aligned} \right. \quad \begin{pmatrix} v_1(0) = 3 \\ v_2(0) = 4
\end{pmatrix} \] \[
  u = \begin{pmatrix} u_1 \\ u_2 \end{pmatrix}, u'(t) = \underbrace{\begin{pmatrix} 1 & 2 \\ 3
  & 4 \end{pmatrix}}{=A} u(t) \Rightarrow e^{tA}
  \]
  EW: \begin{align*}
     \lambda = 5, \quad v_1 \begin{pmatrix} 1 \\ 2 \end{pmatrix} \\ 
     \lambda =  -1 \quad v_2 \begin{pmatrix} -1 \\ 1 \end{pmatrix} \\
     \Rightarrow Y(t) = \begin{pmatrix} e^{5t} & -e^{-t} \\ -e^{5t} & e^{-t}
     \end{pmatrix}
   \end{align*}

   $v_i e^{\lambda_i t}$ bildet ein Fundamentalsystem\footnote{Das
   Fundamentalsystem ist die gesamtheit aller L\"osungen.}
     . $Av_i = \lambda v_i$

  Somit L\"osung $a(t)$ gegeben durch
  \begin{align*}
    u(t) &= Y(t) \begin{pmatrix} cgc_1 \\ c_2 \end{pmatrix} \quad, c_1, c_1 = \text{const
      (\"uber Anfangswerte bestimmt)} \\
      &= \begin{pmatrix} c_1 e^{5t} - c_2 e^{-t} \\ 2c_1e^{5t} + c_2e^{-t}
    \end{pmatrix} \\
  u(0) &= \begin{pmatrix} c-1 - c_2 \\ 2 c_1 + c_2 \end{pmatrix} \mathop{=}^!
  \begin{pmatrix} 3 \\ 4 \end{pmatrix} \\
    \Rightarrow c_1 &= \frac{7}{3}, c_2 = - \frac{2}{3} \\
\Rightarrow u(t) = \begin{pmatrix} u_1(t) \\ u_2(t) \end{pmatrix} =
  \begin{pmatrix} \frac{7}{3} e^{5t} + \frac{2}{3} e^{-t} \\ \frac{14}{3} e^{5t}
  - \frac{2}{3} e^{-t} \end{pmatrix}
  \end{align*}
  \paragraph{Probe}
  \begin{align*}
    u(0) &= \begin{pmatrix} 3 \\ 4 \end{pmatrix} \\
  u'(t) &= \begin{pmatrix} 3 \frac{5}{3} e^{5t} - \frac{2}{3} e^{-t} \\
  \frac{70}{3} e^{5t} + \frac{2}{3} e^{-t} \end{pmatrix} \\
&= \begin{pmatrix} \frac{28}{3} e^{5t} - \frac{4}{3} e^{-t} + \frac{7}{3} e^{5t}
+ \frac{2}{3} e^{-t} \\ \frac{42}{3} e^{5t} - \frac{6}{3} e^{-t} + \frac{28}{3}
e^{5t} + \frac{8}{3} e^{-t} \end{pmatrix} \\
&= \begin{pmatrix} 2 u_2(t) + u_1(t) \\ 3 u_2(t) + 4 u_1(t) \end{pmatrix}
  \end{align*}
\end{Bsp}

\section{Differenzialgleichung spezieller Form}
Eie Differenzialgleichung der Folrm \[ u(t) = f(t) g(u(t)) \]
nennt man eine Differenzialgleichung mit \emph{getrenten Variablen}
\begin{Bsp}
  L\"osen Sie folgendes Anfangswertproblem $f(t) = t, g(u) = 1 + 2 u$ \[
    \left\{ \begin{aligned} u'(t) &= t - (1 + (u(t))^2) \\ u(\sqrt{\frac{\pi}{2}})
                                  &= -1 \end{aligned} \right.  \]
  Schreibe 
  \begin{align*}
    &u' = \frac{\dd u}{\dd t} = t ( 1 + u^2 ) \\
    &\mathop{\Rightarrow}^{\text{T.d.V\footnote{Trennung der Variablen}}} \int
      \frac{1}{1+u^2} \dd u = \int t \dd t \\
      &\Rightarrow \arctan u = \frac{t^2}{2} + c, \text{ $c$
    Integrationskonstante} \\
    &\Rightarrow u(t) = \tan(\frac{t^2}{2} + c) \\
    &\Rightarrow u(\sqrt{\frac{\pi}{2}}) = \tan(\frac{\pi}{4} + c ) = -1, \quad c =
    - \frac{\pi}{2}
  \end{align*}
  Somit \[ u(t) = \tan(\frac{t^2}{2} - \frac{\pi}{2}) \]
  \paragraph{Probe} \[ u'(t) = t(\frac{1}{\cos^2(\frac{t^2}{2} - \frac{\pi}{2})}) \\
    = t ( \frac{\sin^2 + \cos^2}{cos^2}) = t ( \tan^2(\frac{t^2}{2} -
    \frac{\pi}{2}) + 1) \\
    = t (1+ (u(t)^2)) \]
\end{Bsp}

\section{Exacte Differenzialgleichungn}

\begin{Def}
Man nennt eine Differenzialgleichung der Form \[ g(t,u(t)  + h(t,u(t)) u(T) = 0 \]
exakt, falls eine Funktion $F(t,u)$ existiert, sodass $F_t = g, F_u = h$.
\end{Def}

\end{document}
