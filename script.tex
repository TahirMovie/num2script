\documentclass[11pt]{book}
\usepackage{amsfonts}
\usepackage{ulem}
\usepackage{amsmath}
\usepackage{amsthm}
\usepackage{mathtools}
\usepackage{pstricks}
\usepackage{enumitem}
\usepackage[a4paper]{geometry}
\usepackage{fancyhdr}
\usepackage{titlesec}
\usepackage{amssymb}
\usepackage{hyperref}

\newtheorem*{thm}{}

\newcommand{\dd}{\mathrm{d}}
\newcommand{\ee}{\mathrm{e}}
\newcommand{\RR}{\mathbb{R}}
\newcommand{\NN}{\mathbb{N}}
\newcommand{\UU}{\mathcal{U}}

\newcommand{\myTag}[2][]{\tag{#2}\label{#1#2}}
\newcommand{\myQEDEnd}{\hfill\ensuremath{\square}}


%%%%%%%%%%%%%%%%%%%%%%%%%%%%%%%%%%%%%%%%%%%%%%%%%%%%%%%%%%%%%%%%%%%%%%%%%%%%%%%%%%%%
% The folloing styling is largly donated by Robin Lang (uni.robinlang.net)

\geometry{top=2cm, bottom=2.5cm}
\pagestyle{fancy}

\fancyhead[L]{\normalfont\leftmark}

\fancyhead[R]{\color{gray} Numerische Mathematik I}

\definecolor{dblue}{HTML}{003399}
\definecolor{top}{HTML}{096B8F}
\definecolor{code}{HTML}{7C0E78}
\definecolor{dgray}{gray}{0.80}
\definecolor{dkgreen}{rgb}{0,0.6,0}
\definecolor{gray}{rgb}{0.5,0.5,0.5}
\definecolor{mauve}{rgb}{0.58,0,0.82}
\definecolor{dkgray}{gray}{0.33}

\titleformat{\chapter}[hang]
        {\color{dblue}\Huge\bfseries\boldmath\rmfamily}
        {{\fontsize{90pt}{120pt}\selectfont \thechapter}\fontsize{82pt}{100pt}\selectfont\color{dgray}{|}}
        {0em}
        {}
\titlespacing*{\chapter}{0pt}{-10pt}{15pt}

\title{\fontsize{35pt}{55pt}\selectfont\color{dblue} Numerische Mathematik II \\ \ \\ \Large \textbf{Dozent: } Prof. Dr. Christian Rohde}

\author{\color{dkgray}\textbf{Vorlesungsmitschrieb \footnote{Webseite fuer dieses
  Dokument \url{https://github.com/jrapp/num2script}}} \ \\ \small{\color{dkgray}Stand \today}}
\date{\Large\color{dkgray}\textbf{Universit�t Stuttgart, Wintersemester 2013/2014}}

\renewcommand{\thechapter}{\Roman{chapter}}


\theoremstyle{break}
\newtheoremstyle{myStyle} % name of the style to be used
  {}      % measure of space to leave above the theorem. E.g.: 3pt
  {}      % measure of space to leave below the theorem. E.g.: 3pt
  {}        % name of font to use in the body of the theorem
  {}          % measure of space to indent
  {}        % name of head font
  {\newline}        % punctuation between head and body
  {0em}       % space after theorem head; " " = normal interword space
  {\bfseries{#1 \arabic{chapter}.#2} \normalfont[\color{top}#3\color{black}]}        % Manually specify head
\theoremstyle{myStyle}
\newcounter{myCounter}[chapter]
\newtheorem{Def}[myCounter]{Definition}
\newtheorem{Bem}[myCounter]{Bemerkung}
\newtheorem{Bsp}[myCounter]{Beispiel}
\newtheorem{Satz}[myCounter]{Satz}

%%%%%%%%%%%%%%%%%%%%%%%%%%%%%%%%%%%%%%%%%%%%%%%%%%%%%%%%%%%%%%%%%%%%%%%%%%%%%%%%%%%%%%
\begin{document}

\maketitle
\setcounter{tocdepth}{1}
\tableofcontents

\chapter{Gew\"ohnliche Differentialgleichungen: Ein \"Ueberblick}

\section{Grundlegende Defintionen}

F\"uer \( n \in \mathbb{N} \) sei der \uline{Zustandsraum} \( \mathcal{U} \subseteq
\mathbb{R}^m \) gegeben. Weiter sei \( f \in C^0(\mathbb{R} \times \mathcal{U})
\). Au\ss erdem w\"ahle man ein \( t_0 \in \mathbb{R} \) und ein Intervall \( I
\subseteq \mathbb{R} \) mit \( t_0 \in I \). 

Finde \( u = (u_1,\dotsc,u_m)^T \in C^1(I)^m \) mit \[ u = u(t) \in \mathcal{U} fuer t \in I \] und
\begin{align*}
  u'(t) &=f(t,u(t)) \myTag{1.1} \\
  u(t_0) &= v_0 \myTag{1.2}
\end{align*}
\eqref{1.1} hei\ss t \uline{m-dimensionales System ge\"ohnlicher Differentialgleichungen (erster Ordnung)}.
\eqref{1.1},\eqref{1.2} hei\ss t gew\"ohnliche Anfangeswertproblem.

\begin{Def}[Klassische L\"oesung]

  Falls eine Funktion \( u \in C^1(I)^m \) existiert, die \eqref{1.1}, \eqref{1.2} erf\"uellt,
hei\ss t \( u \) \uline{klassische L\"oesung}.

\end{Def}

\begin{Bem}[Spezielle Typen]
  \begin{enumerate}[label=(\roman*)]
    \item \eqref{1.1} hei\ss t linear gdw. \eqref{1.1} von der Form \[ u'(t) = A(t) u(t) + b(t) (t \in \mathbb{R}) \] ist. Dabei sei \( A = A(t) : \mathbb{R} \Rightarrow \mathbb{R}^{n \times m} \) und \( b = b(t): \mathbb{R} \Rightarrow \mathbb{R}^m \).
    \item Die Gleichung \eqref{1.1} hei\ss t autonom, falls \[ f(f,u) = f(u) \] gilt.
  \end{enumerate}

\end{Bem}

\section{Einige Beispiele aus der Modellierung}

\begin{Bsp}[R\"aumliche homogene chemisch aktive Mischung]

Sei \( D \subseteq \mathbb{R}^d, d \in \mathbb{N} \) offen beschrenkt. In \( D \) seien die Stoffe \( A,B, C \) r\"aumlich homogen verteilt.
\begin{align*}
  a &= a[A]  & \text{Konzentration von } &A \text{ in}
  \frac{\text{Mol}}{\text{Volumen}} \\
  b &= b[B]  & &B  \\
  c &= c[C]  & &C 
\end{align*}
Bimollekulare Reaktion \[ 1 A + 1 B \rightharpoonup 1C \myTag[1.3]{*} \] Wir
wollen ein quantitatives Modell f\"ur \eqref{1.3*} herleiten. Gesucht sind 
\begin{align*}
a &= a(x,t) = a(t) \quad \text{(r\"aumliche Homogenit\"at)} \\
  b &= b(x,t) = b(t) \\
  c &= c(x,t) = c(t)
\end{align*}
F\"ur \( t_0 = 0 \) sind die Anfangskonzentrationen \[ a(0) = a_0 \geq 0, \quad
b(0) = b_0 \geq 0, \quad c(0) = c_0 \geq 0 \]
gegeben. W\"ahle \[ u = (a,b,c)  \in U, U = [0,\inf)^3 \]
Es sei \( \Delta t > 0 \) und \( t \in [0,\inf ) \)
\begin{align*}
  a (t+\Delta t) &= a(t) + \text{Reaktionsverlust/-gewinn in} (t,t+\Delta t) \\
                 &= a(t) \int_t^{t +\Delta t} R_A(s) \dd s
\end{align*}
Dabei sei \( R_A(t) \) der Reaktionsverlusst/-gewinn zum Zeitpunkt t. \\
\underline{Konstituive Annahme}:
\[ R_A(t) = - \underbrace{k}_{Reaktionsgeschwindichkeit, > 0}  a(t) b(t) \]
Also: \[ a(t + \Delta t) = a(t) - \int_t^{t + \Delta t} k a(s) b(s) \dd s \]
\uline{Regularit\"atsannahme}: \[ a,b,c \in C^1 ([0,\inf)) \]
Daraus folgt
\begin{align*}
  & & a(t + \Delta t) &= a(t) - \Delta t k a(t) b(t) + \mathcal{O} (\Delta t^2) \\
  &\Leftrightarrow & \frac{a(t+\Delta t) - a(t)}{\Delta t} &= - k a(t) b(t) +
  \mathcal{O}(\Delta t) \\
  &\xRightarrow{\Delta t \rightarrow 0} & a'(t) &= - k a(t) + b(t)
  \myTag[1.3]{**1}
\end{align*}
Mit derselben Argumentation gilt
\[ \begin{aligned}
  b'(t) &= - k a(t) b(t) \\
  c'(t) &= + k a(t) b(t)
\end{aligned} \myTag[1.3]{**} \]
Wie kann man \eqref{1.3**} l\"oesen ?

F\"uer \( S = a -b \) gilt gerade
% TODO: Wie funktioniert align richtig fuer solche sachen?
\begin{align*}
  S'(t) &= a'(t) - b'(t) & &= 0 \\
  \Rightarrow S(t) &= a(t) - b(t) & &=: S_0
\end{align*}
Auswertung bei \( t = 0\) liefert \[ S_0 = a_0 - b_0 \Leftarrow b(t) = a(t) -
S_0 \]
Sei o.B.d.A. \( S_0 \leq 0 (S_0 < 0 \text{ analog}) \). Dann gilt
\begin{align*}
a'(t) = - k a(t) (a(t) - S_0)
\xRightarrow{T.d.V.\footnote{Trennung der Varialen}} a(t) = \frac{a_0 S_0}{a_0 - b_0 e^{-k s_0 t}}
\Rightarrow b(t) = \frac{a_0 S_0}{a_0 - b_0 e^{-k s_0 t}}-S_0
\end{align*}
Au\ss erdem 
\begin{align*}
  c(t) &= c_0 +k \int_0^t a(s) b(s) \dd s \\
       &= c_0 + \int_0^t - b'(t) \dd s \\
       &= c_0 - b(t) + b_0 \\
       &= c_0 + b_0 - b(t)
\end{align*}
\begin{pspicture}
  Fig 1
\end{pspicture}

\end{Bsp}

\begin{Bsp}[CO-Oxidation Platin/Katalyse]
  \begin{align*}
    O_2 + Pt + Pt &\xrightleftharpoons[K_{-1}]{K_1} PtO + PtO \\
    CO + Pt &\rightleftharpoons[K_{-2}]{K_2} PtCO \\
    PtCo + PtO &\xleftharpoonup{K_3} Pt + Pt + CO_2 \\
         CO + Pt &\xrightleftharpoons[K_{-4}]{K_4} \underbrace{[PtCO]}_{\text{Chemisch inaktive Variante}}
  \end{align*}
  \( k_1,k_{-1},k_2,k_{-2},k_3,k_4,k_{-4} > 0 \) \\
  Konzentration: ( \( \in [0,1] \) )
  \begin{align*}
       PtO&: x \quad & CO_2&: 1 \\
      PtCO&: y \quad & CO&: 1 \\
    [PtCO]&: s \quad & O2&: 1\\
      Pt&: z &&
  \end{align*}
  PtO und PtCO werden st\"andig nachgef\"uehrt ( sind in beliebiger konzentration vorhanden)
  \begin{align*}
    x'(t) &=   k_1 z z + k_1 1 z z \\
          &   -k_{-1} x x - L_1 x x \\
          &   -k_3 x x \\
          &=  2 k_1 z^2 - 2 k_{-1} x^2 - k_3 x y \\
    y'(t) &=   k_2 1 z - k_{-2} y - k_3 y x \\
    z'(t) &= -2k_1 z^2 1 + 2 k_{-1} x^2 - k_1 1  z + k_{-2} y \\
          &  +2k_3 x y - k_4 z 1 + k_{-4} s \\
    s'(t) &=   k_4 z - k_{-4} s
  \end{align*}

  Dieses System ist wahlscheinlich nicht eakt l\"oesbar. Es gilt jedoch noch die
  globale Invariante \[ x'(t) + y'(t) + s'(t) + z'(t) = 0 \forall_{t>0} \]

\end{Bsp}

\begin{Bsp}[Verbreitung von Ger\"uchte]
  Sei eine (menschliche) Population der gr\"oe\ss e \( N \in \NN \) gegeben.

  Gesucht: Anzahl der Menschen \( Z= Z(t) \in \UU := [0,N] \), die zum Zeitpunkt
  \( t \geq 0 \) eine bestimmte Nachricht kennen.\\ 
  \uline{Kommunikationsmodell:}
  \begin{enumerate}[label=(\roman*)]
    \item Zum Zeitpunkt \( t_0 = 0 \) kennen \( z_0 \geq 0 \) Menschen die Nachricht.
    \item Die Nachricht wird nur duch Gespr\"ache ``unter vier Augen''
      weitergegeben ( Nachricht = Ger\"uechte).
    \item Jeder Informierte hat in der Zeitspanne \( \Delta t > 0 \) genau \(
      k \Delta t\) Vier-Augen-Kontakte mit anderen Menschen (Informierte und
      nicht-Informierte)
  \end{enumerate}
  Falls wir mit \( q(t) \in [0,1] \) den Anteil der Nichtinformierten
  bezeichnen, gilt \[ q(t) N = N - Z(t) \myTag[1.5]{*} \]
  Da die Zuname von Z im Zentrum \( \Delta t \) kann man durch
  \begin{align*}
    Z(t + \Delta t) & = Z(t) + q k \Delta t Z(t) \\
                    &\overset{\eqref{1.5*}}{=}Z(t) + k \Delta t \frac{N-Z(t)}{N} Z(t)
  \end{align*}
  beschreiben.

  Mit der Annahme \( Z \in C^1([0,\inf],\UU) \) gilt im Limes \( \Delta t \rightarrow 0 \)
  das gew\"ohnliche AWP \[ Z'(t) = k \frac{N-Z(t)}{N} Z(t) , Z(0) = Z_0
  \myTag[1.5]{**} \]

  \begin{pspicture}
    Fig 2: Langzeitverhalten von Z(t)
  \end{pspicture}
  
  Wie kann man \eqref{1.5**} numerisch L\"oesen? Auf der Basis des diskreten Modells kann man den Algorithmus
  \begin{align*}
    Z^{n+1} &= Z^n + \Delta t k \frac{N-Z^n}{N}Z^n, \quad (N \in \NN) \\
    Z^0 &= Z_0
  \end{align*}
  konstruieren. Dabei ist \( Z^n \) eine Approximation \( Z(n \Delta t ) \)
\end{Bsp}

\begin{Bsp}[Bewegunsgleichungen]
  Fuer \( N \in \NN \) betrachten wir die d-dimesionale Bewegung von Masseteilchen \( p_i ( i=1,\dotsc,N) \) mit Masse \( m_i > 0 \)
  Anwendungen: z.B. die Dynamik von Planeten, Sandk\"oernern oder die Molek\"uldynamik.
  Gesucht: Position \( x_i = x_i(t) \in \RR^d, i = 1,\dotsc,N \) des Masseteilchen \( p_i \) zum Zeitpunkt \( t > 0 \)

  Wir schr\"anken uns auf Bewegungen ein, die unter Einfluss einens Potentialfelds \( V:\RR^{dN} \leftarrow \RR \) stattfinden. Falls das Potentialfeld duch ein Gravitationsfeld gegeben ist, gilt z.B.
  \[ V(x_1,\dotsc,x_N) = \sum_{j \neq k, (j,k) = 1,\dotsc,N } \frac{m_j m_k}{|| x_j - x_k||_2} \]
  nach dem Newtischen Gesetz gilt dann \[ m_i x_i''(t) = - \frac{\partial V}{\partial x_i} (x_1(t),\dotsc,x_N(t)) \]
  % Vorlesung 2014-04-16
  Daher ist
  \[ \frac{\partial}{\partial x_i} V(x_1,\dotsc,x_N) = \begin{pmatrix}
        \frac{\partial}{\partial x_i1} V(x_1, \dots, x_N) \\
        \vdots \\
        \frac{\partial}{\partial x_id} V(x_1, \dots, x_N)
  \end{pmatrix} \] 
  F\"uer das Gravitationspotenzial gilt peziell \[ m x_i'')t) = m_i
    \sum_{\stackrel{j=1}{j \neq i}^N} \frac{m_j}{| x_i(t)-x_j(t)|_2^3}
    (x_i(t)-x_j(t)) \]
  Zum Zeitpunkt \( t= 0 \) sind die Anfangspositionen- und -geschwindichkeiten
  durch \[ x_i(0) x^0_i \in \RR^a, x^0_i(0) = v^0_i \in \RR^a \]
  Dabei sein \( x_1^0,\dotsc,x_n^0 \) paarweise verschieden.

  Wir schreiben dieses AWP (zweiter Ordnung) in erin \( (2dN) \) --
  dimensionales System erster Ordnung um. Deibei sei \[ 
    u(t) = \begin{pmatrix} u_1(t) \\ \vdots  \\ 
      u_{2N}(t) \end{pmatrix} = \begin{pmatrix} x_1(t) \\ x'_1(t) \\ x_2(t) \\ x'_2
      \\\vdots  \\
u_{2N}(t) \end{pmatrix} \]

  Damit erhalten wir f\"uer \( u \) das AWP \begin{align*}
    u'(t) &= \begin{pmatrix} u_1(t) \\
        \frac{-1}{m_1} \frac{\partial}{\partial x_1} V(u_1(t),u_3(t), \dotsc, u_{2N-1}(t)) \\
        u_4(t) \\ \vdots \\ 
        \frac{-1}{m_N}\frac{\partial}{\partial x_N} V(u_1(t),u_3(t), \dotsc, u_{2N-1}(t)) 
    \end{pmatrix} \myTag[1.6]{*} \\
    u(0) &= (x_2^0, v_1^0, x_2^0, \dotsc, v_2^0)^T
  \end{align*}
  Damit ist gerade \[ \UU = \{ (x_1,v_1,\dotsc,x_N,v_N)\in\RR^{2dN} |
    x_1,\dotsc,x_N \text{ paarwerise verschieden} \} \]
  Im Graviationsfall sollte \eqref{1.6*} eine klassische L\"oessung auf \(
  I=[0,\infty] \) besitzen (, wenn die physikalische Realit\"ate korret
  beschreibt). \\
  F\"uer \eqref{1.6*} ist die ``nat\"uerliche'' Energie gerade \begin{align*} E(t) =
    \text{kinetische Energie} + \text{Potentialenergie} \\
    &= \frac{1}{2} \sum_{i=1}^N m_i (x_i'(t))^2 + V(x_1(t),\dotsc, x_N(t))
  \end{align*}
  Wir multiplizieren das Newtonsche Kraftgesetz mit \( x_i(t) \)
  \begin{align*}
    m_i\underbrace{x''_i(t) x'_i(t)}{= (\frac{x'_i(t)^2}{2})'} = - x'_i(t)
    \frac{\partial}{\partial x_i} V(x_1(t),\dotsc,x_1(t)) \\
    \Rightarrow \sum_{i=1}^N m_i (\frac{(x'_i(t))^2}{2})' = - \sum_{i=1}^N
    x'_i(t) \frac{\partial}{\partial x_i} V(x_1(t),\dotsc,x_N(t)) \\
    \Rightarrow \sum_{i=1}^N (m_i (\frac{(x'_i(t))^2}{2})' +
    V(x_1(t),\dotsc,x_1(t))' = 0
    \Rightarrow \frac{\dd}{\dd t} E(t) = 0
  \end{align*}
  Falls eine klassische L\"oesung in \( [0, \infty] \) existiert, haben wir \[
    E(t) = E(0) = \sum_{i = 1}^N \frac{1}{2}m_i(v_i^0)^2 + V(x_1^0,\dotsc,x_N^0)
  \]
  \( E \) ist also eine globale Invariante!
  
  \myQEDEnd
\end{Bsp}

\begin{Bsp}[L\"oessungsintervalle]
  Sei \( m = 1 \) und \( u_0 = 0 \). F\"uer \( a \in \RR \) betrachtet das AWP 
  \begin{align*}
    u'(t) = U(t)^2 + a, \quad (t\in I) \\
    u(0) = U_0 = 0
  \end{align*}
  \begin{enumerate}[label=(\alph*)]
    \item \( a = 0 \Rightarrow u(t) = 0 \) auf \( I = \RR \) \\
    \item \( a < 0 \Rightarrow u = u(t) = - \sqrt{-a} \tanh(\sqrt{-1} t) \) auf
      \( I = \RR \) \\
    \item \( a > 0 \Rightarrow u = u(t) = \sqrt{a} \tan(\sqrt{a} t) \) auf \( I
      = (-\frac{\pi}{2\sqrt{a}},\frac{\pi}{2\sqrt{a}}) \subsetneq \RR \)
  \end{enumerate}
\end{Bsp}

\begin{Bsp}[Eindeutigkeit]
  Sei \( m= 1, a_0 = 0 \). Betrachte das AWP \[ u'(t) = \sqrt{u(t)}, u(0)=0 \]
  Es existiert die L\"oesungen \[ u_1(t) = 0, \quad u_2(t) = \frac{t^2}{u},
    \quad u_3(t) = \left\{\begin{aligned} 0, t< 0 \\ \frac{t^2}{4}, t \geq 0
    \end{aligned} \right. \]
    auf ganz \( \RR \)

    \myQEDEnd
\end{Bsp}

F\"uer einen Zusandsraum \( \UU \subseteq \RR^m, u_0 \in \UU, t_0 \in \RR \)
betrachen wir \eqref{1.1},\eqref{1.2}, d.h. \[ u'(t) = f(t,U(t)), u(0) = u_0 \]
Dann gilt
\begin{Satz}[Picard-Lindel\"oef]
  Sei \( a > 0 \) und weiter \( b \geq 0 \) so gew\"awhlt, dass \(
  \overline{B_b(u_0)} \subset \UU \) gilt. Au\ss erdem sei \( L = L(a,b,u_0) > 0 \)
  eine Konstante, sodass alle \(t \in [t_0-a,t_0+a] \) und alle \( w,\tilde{w}
  \in \overline{B_b(u_0)} \) gerade \[ |f(t,w) - f(t,\tilde{w})| \leq L |w -
    \tilde{w}
  | \] gildt (Lipschitz-Stetigkeit in \( [t_0-a,t_0+a] \times \overline{B_b(u_0)} \)
  )
  Dann haben wir
  \begin{enumerate}[label=(\roman*)]
  \item Es existiert eine Zahl \( \alpha \in (0,a], sodass \eqref{1.1}
    \eqref{1.2} \) eine L\"oesung \( u \in C^1(t_0-a,t_0+a,\UU) \) besitzt \\
  \item Es existiert genau eine L\"oesung von \eqref{1.1},\eqref{1.2} in \(
    t_0-a,t_0+a \) \\
  \item F\"uer \( \alpha \) in (i),(ii) gilt in Falle \( \UU = \RR^m, \) \[
      \alpha = \min{a, \frac{b}{A}}, \quad A := \max_{\stackrel{t\in
         [t_0-a,t_0+a]}{w \in \overline{B_b(u_0)}}} \] gilt
  \end{enumerate}
\end{Satz}


\end{document}
